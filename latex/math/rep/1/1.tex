\documentclass[12pt]{article}

\usepackage{amscd}
\usepackage{amsmath}
\usepackage{amssymb}
\usepackage{amsthm}
\usepackage{color}
\usepackage{dsfont}
\usepackage{epsfig}
\usepackage{extarrows}
\usepackage{graphicx}
\usepackage{hyperref}
\usepackage{mathrsfs}
\usepackage{mathtools}
\usepackage{verbatim}
\usepackage{tikz}
\usepackage{tikz-cd} 
\usepackage{xypic}
%\usepackage[all,dvips]{xy}

\usetikzlibrary{matrix}

\begin{comment}  

This LaTeX document is a template to be used by Bates mathematics rising seniors to create a thesis proposal. 

As a guide, the document is already filled out to represent a fictitious proposal, and all you need to do is modify the entries below to represent your own proposal.

A PDF version of the fictitious proposal is available on the department's FAQ and Policies pages, at
                   http://abacus.bates.edu/acad/depts/math/faq.html
      and
                   http://abacus.bates.edu/acad/depts/math/policies.html
      respectively.

Once you have finished your proposal, export it to a PDF file. Give the file a USEFUL name, for example, RiemannThesisProposal.PDF. Email the PDF file to Clementine Brasier, the 
Academic Administrative Assistant for Hathorn Hall, at cbrasier\@bates.edu

                This LaTex document was created Feb/Mar 2010 by Adriana Salerno and updated Feb 2012 by Meredith Greer

\end{comment}

\setlength{\textheight}{8.5in} \setlength{\topmargin}{0.0in}
\setlength{\headheight}{0.0in} \setlength{\headsep}{0.0in}
\setlength{\leftmargin}{0.5in}
\setlength{\oddsidemargin}{0.0in}
%\setlength{\parindent}{1pc}
\setlength{\textwidth}{6.5in}
%\linespread{1.6}

\newtheorem{conjecture}{Conjecture}
\newtheorem*{conjecture*}{Conjecture}

\newtheorem{corollary}{Corollary}
\newtheorem*{corollary*}{Corollary}

\newtheorem{definition}{Definition}
\newtheorem*{definition*}{Definition}

\newtheorem{example}{Example}
\newtheorem*{example*}{Example}

\newtheorem{lemma}{Lemma}
\newtheorem*{lemma*}{Lemma}

\newtheorem{note}{Note}
\newtheorem*{note*}{Note}

\newtheorem{problem}{Problem}
\newtheorem*{problem*}{Problem}

\newtheorem{prop}{Proposition}
\newtheorem*{prop*}{Proposition}

\newtheorem{question}{Question}
\newtheorem*{question*}{Question}

\newtheorem{theorem}{Theorem}
\newtheorem*{theorem*}{Theorem}

%%%%%%%%%%%%%%%%%%%%%%%%%%%%%%%%%%%%%%%%%

\begin{document}

\title{Linear Representations of Finite Groups\\Generalities on Linear Representations}
\author{Yihan Zhang}
\maketitle

%\tableofcontents
\bigskip

Let $V$ be a vector space over $\mathbb C$ of dimension $n$ with basis $\{e_i\}_{i=1}^n$. Denote the general linear group over $V$ by 
\begin{align*}
GL(V)=&\{a:V\to V|a\text{ is an isomorphism}\}\\
=&\left\{(a_{ij})_{n\times n}\in\mathbb C^{n\times n}:a(e_j)=\sum_{i=1}^na_{ij}e_i,\det(a_{ij})\ne 0\right\}.
\end{align*}

Let $G$ be a group of finite order with identity $e$ and multiplication $(s,t)\mapsto st$. A linear representation of $G$ in $V$ is a homomorphism
\begin{align*}
\rho:&G\to GL(V)\\
&s\mapsto \rho(s).
\end{align*}
We call $V$ a representation (space) of $G$, and $\deg\rho=\dim V=n$. It must hold that 
\begin{enumerate}
	\item \[\det\rho(s)\ne0.\]
	\item \[\rho(st)=\rho(s)\rho(t),\text{ i.e., }\rho(st)(i,j)=\sum_{k=1}^n\rho(s)(i,k)\rho(t)(k,j).\]
\end{enumerate}

Let $\rho:G\to GL(V),\rho':G\to GL(V')$ be two representations of same degree. $\rho$ is isomorphic to $\rho'$ if $\exists \tau:V\to V'$ a linear isomorphism, such that $\forall s\in G, \tau\circ \rho(s)=\rho'(s)\circ\tau$, i.e.
\[\begin{tikzcd}
V \arrow{r}{\rho(s)} \arrow[swap]{d}{\tau} & V \arrow{d}{\tau} \\%
V' \arrow{r}{\rho'(s)}& V'
\end{tikzcd}\]
is commutative, or $\exists T\in GL_n(\mathbb C)$, such that $\forall s\in G,T\rho(s)=\rho'(s)T$, or $\rho(s)=T^{-1}\rho'(s)T$.

\begin{example}
Let $\rho:G\to\mathbb C^{\times}$ be a representation of $G$ of degree 1. $\forall s\in G$, it has finite order $m$,
\[1=\rho(1)=\rho(s^m)=\rho(s)^m.\]
Thus $\forall s\in G,|\rho(s)|=1$. In particular, 
\begin{align*}
\mathds1:&G\to\mathbb C^{\times}\\
&s\mapsto1
\end{align*}
is called the unit (trivial) representation.
\end{example}

\begin{example}
Let $G$ be a group of order $g$ and $V$ be a vector space of dimension $g$ with basis $\{e_t\}_{t\in G}$ indexed by elements in $G$. A representation 
\begin{align*}
\rho:&G\to GL(V)\\
&s\mapsto\rho(s)\\
\rho(s):&V\to V\\
&e_t\mapsto e_{st}
\end{align*}
is called a regular representation. Thus $\{\rho(s)(e_1)=e_s:s\in G\}$ forms a basis of $V$.

Conversely, for a representation 
\begin{align*}
\sigma:&G\to GL(W)\\
&s\mapsto\sigma(s),
\end{align*}
if $\{\sigma(s)(w):s\in G\}$ forms a basis of $W$, then $\sigma$ is isomorphic to $\rho$
\end{example}

\begin{example}
Let $G$ act on a finite set $X$, i.e. $\forall s\in G,x\mapsto sx$, such that 
\begin{enumerate}
	\item \[1x=x.\]
	\item \[s(tx)=(st)x,\forall s,t\in G,\forall x\in X.\]
\end{enumerate}
Let $V$ be a vector space of dimension $X$ with basis $\{e_x\}_{x\in X}$. The following representation is called a permutation representation
\begin{align*}
\rho:&G\to GL(V)\\
&s\mapsto\rho(s)\\
\rho(s):&V\to V\\
&e_x\mapsto e_{sx}.
\end{align*}
\end{example}

% \begin{thebibliography}{99}
% NOTE: change the "9" above to "99" if you have MORE THAN 10 references.

% \bibitem{erdos1} Erd\"{o}s, P. (1973). Problems and results on combinatorial number theory \uppercase\expandafter{\romannumeral 1}. In \textit{A survey of combinatorial theory (Proc. Internat. Sympos., Colorado State Univ., Fort Collins, Colo., 1971)} (pp. 117-138).

% \end{thebibliography}

%%%%%%%%%%%%%%%%%%%%%%%%%%%%%%%%%%%%%%%%%

\end{document} 
