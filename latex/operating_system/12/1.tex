\documentclass[11pt]{article}

\usepackage{abstract}
\usepackage{algorithm}
\usepackage{algorithmic}
\usepackage{amsmath}
\usepackage{amssymb}
\usepackage{bm}
\usepackage{caption}
\usepackage{CJKutf8}
\usepackage{color}
\usepackage{enumitem}
\usepackage{epsfig}
\usepackage{fancyhdr}
\usepackage{float}
\usepackage{graphics}
\usepackage{graphicx}
\usepackage{geometry}
\usepackage{indentfirst}
\usepackage{lastpage}
\usepackage{listings}
\usepackage{mathdots}
\usepackage{mathpazo}
\usepackage{multirow}
\usepackage{pstricks-add}
\usepackage{pst-blur}
\usepackage{subcaption}
\usepackage{tikz}
\usepackage{wasysym}
\usepackage{xcolor}
\usepackage[BoldFont,SlantFont,CJKsetspaces,CJKchecksingle]{xeCJK}

\allowdisplaybreaks
\DeclareMathOperator*{\argmin}{argmin}
\definecolor{Blue}{rgb}{1.,0.75,0.8}
\definecolor{mygray}{rgb}{0.5,0.5,0.5}
\definecolor{mygreen}{rgb}{0,0.6,0}
\definecolor{mymauve}{rgb}{0.58,0,0.82}
\pagestyle{empty}
\parindent 2em   %段首缩进
\setlength{\parindent}{2em}
\setCJKmainfont[BoldFont=SimHei]{SimSun}
\setCJKmonofont{SimSun}% 设置缺省中文字体
\usetikzlibrary{arrows, automata, calc, shapes}

\newcommand{\HRule}{\rule{\linewidth}{0.5mm}}
\newcommand{\hytt}[1]{\texttt{\hyphenchar\font=\defaulthyphenchar #1}}
\renewcommand{\algorithmicrequire}{\textbf{Input:}}   
\renewcommand{\algorithmicensure}{\textbf{Output:}}  
% \hyphenation{read-Sym-bol re-ad-Space-Tab-New-line str-Tab}

%\footnotesize
\lstset{ %
  backgroundcolor=\color{white},   % choose the background color; you must add \usepackage{color} or \usepackage{xcolor}
  basicstyle=\ttfamily,            % the size of the fonts that are used for the code
  breakatwhitespace=false,         % sets if automatic breaks should only happen at whitespace
  breaklines=true,                 % sets automatic line breaking
  captionpos=b,                    % sets the caption-position to bottom
  commentstyle=\ttfamily\color{mygreen},    
                                   % comment style
  deletekeywords={},               % if you want to delete keywords from the given language
  escapeinside={},                 % if you want to add LaTeX within your code
  extendedchars=true,              % lets you use non-ASCII characters; for 8-bits encodings only, does not work with UTF-8
  frame=single,                    % adds a frame around the code
  keepspaces=true,                 % keeps spaces in text, useful for keeping indentation of code (possibly needs columns=flexible)
  keywordstyle=\color{blue},       % keyword style
  language=C++,                    % the language of the code
  morekeywords={},                 % if you want to add more keywords to the set
  numbers=left,                    % where to put the line-numbers; possible values are (none, left, right)
  numbersep=5pt,                   % how far the line-numbers are from the code
  numberstyle=\tiny\color{mygray}, % the style that is used for the line-numbers
  rulecolor=\color{black},         % if not set, the frame-color may be changed on line-breaks within not-black text (e.g. comments (green here))
  showspaces=false,                % show spaces everywhere adding particular underscores; it overrides 'showstringspaces'
  showstringspaces=false,          % underline spaces within strings only
  showtabs=false,                  % show tabs within strings adding particular underscores
  stepnumber=1,                    % the step between two line-numbers. If it's 1, each line will be numbered
  stringstyle=\color{mymauve},     % string literal style
  tabsize=2,                       % sets default tabsize to 2 spaces
  title=\lstname                   % show the filename of files included with \lstinputlisting; also try caption instead of title
}

\pagestyle{fancy}
\rhead{page \thepage\ of \pageref{LastPage}}
%\chead{}
\lhead{操作系统实验报告}
\cfoot{}

\begin{document}

\title{操作系统实验1 \quad 熟悉Linux系统}
\author{计算机1202 \quad 张艺瀚\\学号:20123852}
\maketitle

\thispagestyle{fancy}
%\newpage
\normalsize

\section{题目}
熟悉Linux系统的基本操作和开发环境

\section{目的}
熟悉和掌握Linux基本命令,熟悉Linux编程环境,为以后的实验做好准备。

\section{要求}
\begin{enumerate}
\item 熟练掌握Linux基本文件命令。
\item 掌握Linux编辑程序、对源代码进行编译、连接、运行及调试的过程。
\item 认真做好预习,书写预习报告。
\item 实验完成后要认真总结、完成实验报告。
\end{enumerate}

\section{常用Linux命令及含义}
见表\ref{tab: linux}

\begin{table}[htbp]
\centering  % 表居中
\begin{tabular}{ll}  % {lccc} 表示各列元素对齐方式,left-l,right-r,center-c
\hline
命令 & 用法 \\
\hline  % \hline 在此行下面画一横线
创建子目录命令 \texttt{mkdir} & \texttt{mkdir subdir} \\
切换路径命令 \texttt{cd} & \texttt{cd subdir} \\
显示当前工作路径命令 \texttt{pwd} & \texttt{pwd} \\
拷贝文件命令 \texttt{cp} & \texttt{cp sfile dfile} \\
删除文件命令 \texttt{rm} & \texttt{rm filename} \\
删除目录命令 \texttt{rmdir} & \texttt{rmdir sdir(子目录为空目录)} \\
显示文件内容或把file1与file2的连接 \texttt{cat} & 显示文件内容 \texttt{cat filename} \\
                                            & 把file1与file2的连接 \texttt{cat file1 file2 > file3} \\
改变文件或目录的防问权限 \texttt{chmod -rwxr-x--x} & \texttt{chmod +/- 数值 filename} \\
改变文件或目录的拥有者 \texttt{chown} & \texttt{chown filename username} \\
列表显示文件名或目录名 \texttt{ls} & \texttt{ls} 或 \texttt{ls -l} \\
查找文件或目录 \texttt{find} & \texttt{find 文件或目录}  \\
显示进程或状态 \texttt{ps} & \texttt{ps} \\
杀进程命令 \texttt{kill} & \texttt{kill pid} \\
编译 & \texttt{gcc -o 目标文件 源文件(.c)} \\
    & \texttt{g++ -o 目标文件 源文件(.cpp)} \\
运行 & \texttt{./目标文件} \\
保存运行结果 & \texttt{./目标文件>文件名} \\
显示运行结果 & \texttt{cat 文件名} \\
移动相对路径下的文件到绝对路径下 & \texttt{mv 路径/文件 路径/文件} \\
在当前目录下改名 & \texttt{mv 文件名 新名称} \\
编译 & \texttt{make} \\
安装编译好的源码包 & \texttt{make install} \\
安装包 & \texttt{sudo apt-get install package} \\
删除包 & \texttt{sudo apt-get remove package} \\
更新源 & \texttt{sudo apt-get update} \\
更新已安装的包 & \texttt{sudo apt-get upgrade} \\
升级系统 & \texttt{sudo apt-get dist-upgrade} \\
使用dselect升级 & \texttt{sudo apt-get dselect-upgrade} \\
安装相关的编译环境 & \texttt{sudo apt-get build-dep package} \\
下载该包的源代码 & \texttt{apt-get source package} \\
检查是否有损坏的依赖 & \texttt{sudo apt-get check} \\
清理所有软件缓存 & \texttt{sudo apt-get clean} \\
\hline
\end{tabular}
\caption{常用Linux命令及含义\label{tab: linux}}
\end{table}

\section{在Linux环境下编制、调试源程序的过程}
在Ubuntu 120.4 LTS下编制C++程序。有源文件\texttt{thread.cpp},\texttt{pcb.h},\texttt{pcb.cpp},\texttt{group\_tree.h},\texttt{group\_tree.cpp}。它们的include关系如下(见代码清单\ref{lst: thread_cpp}-\ref{lst: group_tree_cpp}):

\begin{center}
\begin{lstlisting}[caption = {\texttt{thread.cpp}的include关系}, label = {lst: thread_cpp}]
// thread.cpp
#include "pcb.h"
#include "group_tree.h"
\end{lstlisting}
\end{center}

\begin{center}
\begin{lstlisting}[caption = {\texttt{pcb.h}的include关系}, label = {lst: pcb_h}]
// pcb.h
#ifndef PCB_H
#define PCB_H
#endif // PCB_H
\end{lstlisting}
\end{center}

\begin{center}
\begin{lstlisting}[caption = {\texttt{pcb.cpp}的include关系}, label = {lst: pcb_cpp}]
// pcb.cpp
#include "pcb.h"
\end{lstlisting}
\end{center}

\begin{center}
\begin{lstlisting}[caption = {\texttt{group\_tree.h}的include关系}, label = {lst: group_tree_h}]
// group_tree.h
#ifndef GROUP_TREE_H
#define GROUP_TREE_H
#include "pcb.h"
#endif // GROUP_TREE_H
\end{lstlisting}
\end{center}

\begin{center}
\begin{lstlisting}[caption = {\texttt{group\_tree.cpp}的include关系}, label = {lst: group_tree_cpp}]
// group_tree.cpp
#include "group_tree.h"
\end{lstlisting}
\end{center}

编写\texttt{makefile}如下:
\begin{center}
\begin{lstlisting}[caption = {\texttt{makefile}代码清单}, label = {lst: makefile}]
thread: thread.o pcb.o group_tree.o
  g++ -std=c++14 -o thread thread.o pcb.o group_tree.o
thread.o: thread.cpp pcb.h group_tree.h
  g++ -std=c++14 -c thread.cpp
pcb.o: pcb.cpp pcb.h
  g++ -std=c++14 -c pcb.cpp
group_tree.o: group_tree.cpp group_tree.h pcb.h
  g++ -std=c++14 -c group_tree.cpp
\end{lstlisting}
\end{center}

在终端中\texttt{cd}到当前目录下,运行\texttt{make}编译源代码,显示如下信息:
\begin{center}
\begin{figure}[htbp]
\includegraphics[width=\textwidth]{os-12-1.png}
\caption{编译通过效果图}
\label{fig: compile}
\end{figure}
\end{center}

说明编译通过。输入\texttt{./thread}运行可执行文件。

\end{document}
