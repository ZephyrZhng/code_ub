\documentclass[11pt]{article}

\usepackage{abstract}
\usepackage{algorithm}
\usepackage{algorithmic}
\usepackage{amsmath}
\usepackage{amssymb}
\usepackage{bm}
\usepackage{caption}
\usepackage{CJKutf8}
\usepackage{color}
\usepackage{enumitem}
\usepackage{epsfig}
\usepackage{fancyhdr}
\usepackage{float}
\usepackage{graphics}
\usepackage{graphicx}
\usepackage{geometry}
\usepackage{indentfirst}
\usepackage{lastpage}
\usepackage{listings}
\usepackage{mathdots}
\usepackage{mathpazo}
\usepackage{multirow}
\usepackage{pstricks-add}
\usepackage{pst-blur}
\usepackage{subcaption}
\usepackage{tikz}
\usepackage{wasysym}
\usepackage{xcolor}
\usepackage[BoldFont,SlantFont,CJKsetspaces,CJKchecksingle]{xeCJK}

\allowdisplaybreaks
\DeclareMathOperator*{\argmin}{argmin}
\definecolor{Blue}{rgb}{1.,0.75,0.8}
\definecolor{mygray}{rgb}{0.5,0.5,0.5}
\definecolor{mygreen}{rgb}{0,0.6,0}
\definecolor{mymauve}{rgb}{0.58,0,0.82}
\pagestyle{empty}
\parindent 2em   %段首缩进
\setlength{\parindent}{2em}
\setCJKmainfont[BoldFont=SimHei]{SimSun}
\setCJKmonofont{SimSun}% 设置缺省中文字体
\usetikzlibrary{arrows, automata, calc, shapes}

\newcommand{\HRule}{\rule{\linewidth}{0.5mm}}
\newcommand{\hytt}[1]{\texttt{\hyphenchar\font=\defaulthyphenchar #1}}
\renewcommand{\algorithmicrequire}{\textbf{Input:}}   
\renewcommand{\algorithmicensure}{\textbf{Output:}}  
% \hyphenation{read-Sym-bol re-ad-Space-Tab-New-line str-Tab}

%\footnotesize
\lstset{ %
  backgroundcolor=\color{white},   % choose the background color; you must add \usepackage{color} or \usepackage{xcolor}
  basicstyle=\ttfamily,            % the size of the fonts that are used for the code
  breakatwhitespace=false,         % sets if automatic breaks should only happen at whitespace
  breaklines=true,                 % sets automatic line breaking
  captionpos=b,                    % sets the caption-position to bottom
  commentstyle=\ttfamily\color{mygreen},    
                                   % comment style
  deletekeywords={},               % if you want to delete keywords from the given language
  escapeinside={},                 % if you want to add LaTeX within your code
  extendedchars=true,              % lets you use non-ASCII characters; for 8-bits encodings only, does not work with UTF-8
  frame=single,                    % adds a frame around the code
  keepspaces=true,                 % keeps spaces in text, useful for keeping indentation of code (possibly needs columns=flexible)
  keywordstyle=\color{blue},       % keyword style
  language=C++,                    % the language of the code
  morekeywords={},                 % if you want to add more keywords to the set
  numbers=left,                    % where to put the line-numbers; possible values are (none, left, right)
  numbersep=5pt,                   % how far the line-numbers are from the code
  numberstyle=\tiny\color{mygray}, % the style that is used for the line-numbers
  rulecolor=\color{black},         % if not set, the frame-color may be changed on line-breaks within not-black text (e.g. comments (green here))
  showspaces=false,                % show spaces everywhere adding particular underscores; it overrides 'showstringspaces'
  showstringspaces=false,          % underline spaces within strings only
  showtabs=false,                  % show tabs within strings adding particular underscores
  stepnumber=1,                    % the step between two line-numbers. If it's 1, each line will be numbered
  stringstyle=\color{mymauve},     % string literal style
  tabsize=2,                       % sets default tabsize to 2 spaces
  title=\lstname                   % show the filename of files included with \lstinputlisting; also try caption instead of title
}

\pagestyle{fancy}
\rhead{page \thepage\ of \pageref{LastPage}}
%\chead{}
\lhead{操作系统实验报告}
\cfoot{}

\begin{document}

\title{操作系统实验3 \quad 进程同步和通信}
\author{计算机1202 \quad 张艺瀚\\学号:20123852}
\maketitle

\thispagestyle{fancy}
%\newpage
\normalsize 

\section{实验题目}
生产者和消费者问题模拟

\section{实验目的}
这是一个验证型实验。通过对给出的程序进行验证、修改,进一步加深理解进程的概念,了解同步和通信的过程,掌握进程通信和同步的机制,特别是利用缓冲区进行同步和通信的过程。通过补充新功能,加强对知识的灵活运用,培养创新能力。

\section{实验要求}
\begin{enumerate}
  \item 调试、运行给出的程序,从操作系统原理的角度验证程序的正确性。
  \item 发现并修改程序中的原理性错误或不完善的地方。 
  \item 鼓励在程序中增加新的功能。完成基本。
  \item 在程序中适当地加入注释。
  \item 认真进行预习,阅读原程序,发现其中的原理性错误,完成预习报告。
  \item 实验完成后,要认真总结,完成实验报告。
\end{enumerate}

\section{程序流程图}
见图\ref{fig: main}-\ref{fig: runc}

\tikzstyle{startstop} = [rectangle, rounded corners, minimum width=2cm, minimum height=1cm,text centered, draw=black, fill=red!30]
\tikzstyle{io} = [trapezium, trapezium left angle=70, trapezium right angle=110, minimum width=2cm, minimum height=1cm, text centered, draw=black, fill=blue!30]
\tikzstyle{operation} = [rectangle, minimum width=2cm, minimum height=1cm, text centered, draw=black, fill=orange!30]
\tikzstyle{judge} = [diamond, minimum width=2cm, minimum height=1cm, text centered, draw=black, fill=green!30]
\tikzstyle{arrow} = [thick,->,>=stealth]

\begin{figure}[htbp]
\centering
\begin{tikzpicture}[node distance=2cm]
%定义流程图具体形状
\node (a) [startstop] at(0,-1)       {开始};
\node (b) [operation] at(0,-2.5)      {初始化};
\node (c) [operation] at(0,-4)      {输入};
\node (d) [judge] at(0,-6)      {命令?};

\node (e) [operation] at(0,-8)      {runp};
\node (f) [judge] at(0,-10)      {状态?};
\node (g) [operation] at(0,-12.5)      {生产者阻塞};
\node (h) [judge] at(0,-17)      {生产者wait?};
\node (i) [operation] at(0,-20)      {生产者计数+1};

\node (p) [operation] at(-3,-11)      {唤醒消费者};
\node (q) [operation] at(-3,-12.5)      {runc};
\node (r) [operation] at(-3,-14)      {消费者计数-1};

\node (k) [operation] at(3.2,-8)      {runc};
\node (l) [judge] at(3.2,-10)      {状态?};
\node (m) [operation] at(3.2,-12.5)      {消费者阻塞};
\node (n) [judge] at(3.2,-17)      {消费者wait?};
\node (o) [operation] at(3.2,-20)      {消费者计数+1};

\node (s) [operation] at(6.2,-11)      {唤醒生产者};
\node (t) [operation] at(6.2,-12.5)      {runp};
\node (u) [operation] at(6.2,-14)      {生产者计数-1};

\node (j) [startstop] at(0,-22)      {结束};

%连接具体形状
\draw [arrow](a) -- (b);
\draw [arrow](b) -- (c);
\draw [arrow](c) -- (d);

\draw [arrow](d.west) -- (-5, -6) -- node[anchor=south] {'e'} (-1, -6) -| (-5, -21.2) -- (0, -21.2) -- (j.north);

\draw [arrow](d) -- node[anchor=west] {'p'} (0, -7.2) -- (e);
\draw [arrow](e) -- (f);
\draw [arrow](f) -- node[anchor=east] {sleep} (0, -11.5) -- (g);
\draw [arrow](g) -- (h);
\draw [arrow](h) -- node[anchor=east] {Y} (0, -19.3) -- (i);

\draw [arrow](f.west) -- node[anchor=south] {awake} (-3, -10) -- (-3, -10) -- (p.north);
\draw [arrow](p) -- (q);
\draw [arrow](q) -- (r);
\draw [arrow](r) -- (-3, -15) -- (0, -15);

\draw [arrow](d.east) -- (3.2, -6) -- node[anchor=west] {'c'} (3.2, -7.2) -- (k.north);
\draw [arrow](k) -- (l);
\draw [arrow](l) -- node[anchor=west] {sleep} (3.2, -11.5) -- (m);
\draw [arrow](m) -- (n);
\draw [arrow](n) -- node[anchor=west] {Y} (3.2, -19.3) -- (o);

\draw [arrow](l.east) -- node[anchor=south] {awake} (6.2, -10) -- (6.2, -10) -- (s.north);
\draw [arrow](s) -- (t);
\draw [arrow](t) -- (u);
\draw [arrow](u) -- (6.2, -15) -- (3.2, -15);

\draw [arrow](i) -- (0, -20.9) -- (9.2, -20.9) -- (9.2, -3.25) -- (0, -3.25);
\draw [arrow](o) -- (3.2, -20.9);
\draw [arrow](h.west) -- node[anchor=south] {N} (-3, -17) -- (-3, -17) -- (-3, -20.9) -- (0, -20.9);
\draw [arrow](n.east) -- node[anchor=south] {N} (6.2, -17) -- (6.2, -17) -- (6.2, -20.9);
%\draw [arrow]( ) -- node[anchor=east] {是} ( );
%\draw [arrow]( ) -- node[anchor=south] {否} ( );
\end{tikzpicture}
\caption{主过程}
\label{fig: main}
\end{figure}

\begin{figure}[htbp]
\centering
\begin{tikzpicture}[node distance=2cm]
%定义流程图具体形状
\node (a) [startstop] at(0,-0.35)       {入口};
\node (b) [operation] at(0,-1.85)      {修改状态为run};
\node (c) [judge] at(0,-3.85)      {管道满?};
\node (d) [operation] at(0,-6)      {数据送管道};
\node (e) [operation] at(0,-7.5)      {修改写指针};
\node (f) [operation] at(0,-9)      {数据计数+1};
\node (g) [operation] at(0,-10.5)      {生产者运行次数+1};
\node (h) [operation] at(0,-12)      {修改状态为ready};
\node (i) [judge] at(0,-15)      {\begin{tabular}{c} 消费者等待 \\ 队列不空? \end{tabular}};

\node (k) [operation] at(3,-6)      {修改状态为wait};
\node (l) [startstop] at(3,-7.5)      {返回sleep};

\node (m) [startstop] at(-3,-18)      {返回normal};

\node (j) [startstop] at(0,-18)      {返回awake};

%连接具体形状
\draw [arrow](a) -- (b);
\draw [arrow](b) -- (c);
\draw [arrow](c) -- node[anchor=east] {N} (0, -5.2) -- (d);
\draw [arrow](d) -- (e);
\draw [arrow](e) -- (f);
\draw [arrow](f) -- (g);
\draw [arrow](g) -- (h);
\draw [arrow](h) -- (i);
\draw [arrow](i) -- node[anchor=west] {Y} (0, -17.2) -- (j);

\draw [arrow](c) -- node[anchor=south] {Y} (3, -3.85) -- (3, -3.85) -- (k);
\draw [arrow](k) -- (l);

\draw [arrow](i) -- node[anchor=south] {N} (-3, -15) -- (-3, -15) -- (m);

%\draw [arrow]( ) -- node[anchor=east] {是} ( );
%\draw [arrow]( ) -- node[anchor=south] {否} ( );
\end{tikzpicture}
\caption{runp}
\label{fig: runp}
\end{figure}

\begin{figure}[htbp]
\centering
\begin{tikzpicture}[node distance=2cm]
%定义流程图具体形状
\node (a) [startstop] at(0,-0.35)       {入口};
\node (b) [operation] at(0,-1.85)      {修改状态为run};
\node (c) [judge] at(0,-3.85)      {管道空?};
\node (d) [operation] at(0,-6)      {取数据};
\node (e) [operation] at(0,-7.5)      {修改读指针};
\node (f) [operation] at(0,-9)      {数据计数-1};
\node (g) [operation] at(0,-10.5)      {消费者运行次数+1};
\node (h) [operation] at(0,-12)      {修改状态为ready};
\node (i) [judge] at(0,-15)      {\begin{tabular}{c} 生产者等待 \\ 队列不空? \end{tabular}};

\node (k) [operation] at(3,-6)      {修改状态为wait};
\node (l) [startstop] at(3,-7.5)      {返回sleep};

\node (m) [startstop] at(-3,-18)      {返回normal};

\node (j) [startstop] at(0,-18)      {返回awake};

%连接具体形状
\draw [arrow](a) -- (b);
\draw [arrow](b) -- (c);
\draw [arrow](c) -- node[anchor=east] {N} (0, -5.2) -- (d);
\draw [arrow](d) -- (e);
\draw [arrow](e) -- (f);
\draw [arrow](f) -- (g);
\draw [arrow](g) -- (h);
\draw [arrow](h) -- (i);
\draw [arrow](i) -- node[anchor=west] {Y} (0, -17.2) -- (j);

\draw [arrow](c) -- node[anchor=south] {Y} (3, -3.85) -- (3, -3.85) -- (k);
\draw [arrow](k) -- (l);

\draw [arrow](i) -- node[anchor=south] {N} (-3, -15) -- (-3, -15) -- (m);

%\draw [arrow]( ) -- node[anchor=east] {是} ( );
%\draw [arrow]( ) -- node[anchor=south] {否} ( );
\end{tikzpicture}
\caption{runc}
\label{fig: runc}
\end{figure}

\section{源程序}
增加生产者计数和消费者计数,作用相当与生产者等待队列和消费者等待队列。若有等待消费者,则生产的数据不会出现在队列中,知道满足所有等待消费者,讲述写入上次指针位置。

\begin{lstlisting}[caption = {代码清单}, label = {lst: code}]
/***************************************************************/
/*         PROGRAM NAME:         PRODUCER_CONSUMER             */
/*    This program simulates two processes, producer which     */
/* continues  to produce message and  put it into a buffer     */
/* [implemented by PIPE], and consumer which continues to get  */
/* message from the buffer and use it.                         */
/*    The program also demonstrates the synchronism between    */
/* processes and uses of PIPE.                                 */
/***************************************************************/

#define PIPESIZE 8

#define PRODUCER 0
#define CONSUMER 1

#define RUN      0    /* statu of process */
#define WAIT     1    /* statu of process */
#define READY    2    /* statu of process */

#define NORMAL   0
#define SLEEP    1
#define AWAKE    2

#include <stdio.h>
struct pcb
{
  char *name;
  int  statu;
  int  time;  /* times of execution */
};

struct pipetype
{
  char type;
  int  writeptr;
  int  readptr;
  struct pcb *pointp; /* write wait point */ // producer queue
  struct pcb *pointc; /* read wait point  */ // consumer queue
};

int pipe[PIPESIZE];
struct pipetype pipetb;
struct pcb process[2];
int count=0;  // resource count
int producerCount=0, consumerCount=0; // semaphore

int runp(int out, struct pcb p[], int pipe[], struct pipetype *tb, int t)
{
  p[t].statu=RUN;
  printf("run PRODUCER. product %d     ",out);
  if(count >= 8)  // full
  {
    p[t].statu=WAIT;
    return(SLEEP);
  }
  pipe[tb->writeptr]=out;
  tb->writeptr=++tb->writeptr%8;  // revised
  // tb->writeptr++;
  printf("writeptr%d\n",tb->writeptr);  // revised
  count++;  // revised
  printf("count = %d\n",count); // revised
  p[t].time++;
  printf("time = %d\n",p[t].time);  // revised
  p[t].statu=READY;
  if((tb->pointc) != NULL)
  {
    printf("return AWAKE");
    return(AWAKE);
  }
  return(NORMAL);
}

int runc(struct pcb p[], int pipe[], struct pipetype *tb, int t)
{
  int c;
  p[t].statu = RUN;
  printf("run CONSUMER. ");
  if(count <= 0)  //
  {
    p[t].statu=WAIT;
    return(SLEEP);
  }
  c=pipe[tb->readptr];
  pipe[tb->readptr]=0; // revised
  // tb->readptr++;
  tb->readptr=++tb->readptr%8;  // revised
  printf("readptr=%d\n",tb->readptr); // revised
  printf(" use %d      ",c);
  // delete some code
  // if(tb->readptr >= tb->writeptr)
  // {
  //  tb->readptr=tb->writeptr=0;
  // }
  count--;// revised
  printf("count = %d\n",count); // revised
  p[t].time++; 
  printf("time = %d\n",p[t].time);  // revised
  p[t].statu=READY;
  if(tb->pointp != NULL)
  {
    printf("return AWAKE\n"); // revised
    return(AWAKE);
  }
  return(NORMAL);
}

void prn(struct pcb p[], int pipe[], struct pipetype tb)
{
  int i;
  printf("\n         ");
  for(i=0;i<PIPESIZE;i++)
  {
    printf("------ ");  // print upper border
  }
  printf("\n        |");  // left dash
  for(i=0;i<PIPESIZE;i++) // second line and vertical bar
  {
    if(pipe[i] != 0)
    {
      printf("  %2d  |",pipe[i]);
    }
    else
    {
      printf("      |");
    }
  }
  printf("\n         ");
  for(i=0;i<PIPESIZE;i++) // print lower border
  {
    printf("------ ");
  }
  printf("\nwriteptr = %d, readptr = %d,  ",tb.writeptr,tb.readptr);
  if(p[PRODUCER].statu == WAIT)
  {
    printf("PRODUCER wait ");
  }
  else
  {
    printf("PRODUCER ready ");
  }
  if(p[CONSUMER].statu == WAIT)
  {
    printf("CONSUMER wait ");
  }
  else
  {
    printf("CONSUMER ready ");
  }
  printf("\n");
}

int main(int argc, char** argv)
{
  int output,ret,i;
  char in[2];

  pipetb.type = 'c';
  pipetb.writeptr = 0;
  pipetb.readptr = 0;
  pipetb.pointp = pipetb.pointc = NULL;
  process[PRODUCER].name = "Producer\0";
  process[CONSUMER].name = "Consumer\0";
  process[PRODUCER].statu = process[CONSUMER].statu = READY;
  process[PRODUCER].time = process[CONSUMER].time = 0;
  output = 0;
  printf("Now starting the program!\n");
  printf("Press 'p' to run PRODUCER, press 'c' to run CONSUMER.\n");
  printf("Press 'e' to exit from the program.\n");
  for(i=0;i<1000;i++)
  {
    in[0]='N';
    while(in[0] == 'N')
    {
      scanf("%s",in);
      if(in[0] != 'e'&&in[0] != 'p'&&in[0] != 'c')  // if input is not e/p/c, input again
      {
        in[0]='N';
      }
    }
    if(in[0] == 'e')
    {
      return 1; // exit(1);
    }
    if(in[0] == 'p'&&process[PRODUCER].statu == READY)
    {
      if(count<8) // revised
      {
        output = (output+1)%100;
      }
      if((ret=runp(output,process,pipe,&pipetb,PRODUCER)) == SLEEP)
      {
        pipetb.pointp = &process[PRODUCER];
      }
      if(ret == AWAKE)
      {
        (pipetb.pointc)->statu=READY;
        runc(process,pipe,&pipetb,CONSUMER);  // revised
        consumerCount--;  // revised
        printf("consumerCount=%d\n",consumerCount); // revised
        if(consumerCount == 0)  // revised
        { // revised
          pipetb.pointc=NULL;// revised
        } // revised
      }
    }
    if(in[0] == 'c'&&process[CONSUMER].statu == READY)
    {
      if((ret=runc(process,pipe,&pipetb,CONSUMER)) == SLEEP)
      {
        pipetb.pointc = &process[CONSUMER];
      }
      if(ret == AWAKE)
      {
        (pipetb.pointp)->statu=READY;
        output=(output+1)%100;  // revised
        runp(output,process,pipe,&pipetb,PRODUCER);
        producerCount--;  // revised
        printf("producerCount=%d\n",producerCount); // revised
        if(producerCount == 0)  // revised
        { // revised
          pipetb.pointp=NULL;// revised
        } // revised
      }
    }
    if(in[0] == 'p'&&process[PRODUCER].statu == WAIT)
    {
      producerCount++;  // revised
      printf("producerCount=%d\n",producerCount); // revised
      printf("PRODUCER is waiting, can't be scheduled.\n");
    }
    if(in[0] == 'c'&&process[CONSUMER].statu == WAIT)
    {
      consumerCount++;  // revised
      printf("consumerCount=%d\n",consumerCount); // revised
      printf("CONSUMER is waiting, can't be scheduled.\n");
    }
    prn(process,pipe,pipetb); 
    in[0]='N';
  }
  return 0;
}
\end{lstlisting}

\section{运行结果及其说明}
队列满后继续生产,生产者阻塞,且修改生产者计数;队列空后继续消费,消费者阻塞,且修改消费者计数。生产者生产数据后,若有等待的消费者,则唤醒,否则数据送管道。

\begin{center}
  \centering
  \includegraphics[width=0.4\textwidth]{3-1.png} 
  \includegraphics[width=0.4\textwidth]{3-2.png} 
  \includegraphics[width=0.4\textwidth]{3-3.png} 
  \includegraphics[width=0.4\textwidth]{3-4.png} 
  \includegraphics[width=0.4\textwidth]{3-5.png} 
  \includegraphics[width=0.4\textwidth]{3-6.png} 
  \includegraphics[width=0.4\textwidth]{3-7.png} 
\end{center}

\end{document}
