\documentclass[11pt]{article}

\usepackage{abstract}
\usepackage{algorithm}
\usepackage{algorithmic}
\usepackage{amsmath}
\usepackage{amssymb}
\usepackage{bm}
\usepackage{caption}
\usepackage{CJKutf8}
\usepackage{color}
\usepackage{enumitem}
\usepackage{epsfig}
\usepackage{fancyhdr}
\usepackage{float}
\usepackage{graphics}
\usepackage{graphicx}
\usepackage{geometry}
\usepackage{indentfirst}
\usepackage{lastpage}
\usepackage{listings}
\usepackage{mathdots}
\usepackage{mathpazo}
\usepackage{multirow}
\usepackage{pstricks-add}
\usepackage{pst-blur}
\usepackage{subcaption}
\usepackage{tikz}
\usepackage{wasysym}
\usepackage{xcolor}
\usepackage[BoldFont,SlantFont,CJKsetspaces,CJKchecksingle]{xeCJK}

\allowdisplaybreaks
\DeclareMathOperator*{\argmin}{argmin}
\definecolor{Blue}{rgb}{1.,0.75,0.8}
\definecolor{mygray}{rgb}{0.5,0.5,0.5}
\definecolor{mygreen}{rgb}{0,0.6,0}
\definecolor{mymauve}{rgb}{0.58,0,0.82}
\pagestyle{empty}
\parindent 2em   %段首缩进
\setlength{\parindent}{2em}
\setCJKmainfont[BoldFont=SimHei]{SimSun}
\setCJKmonofont{SimSun}% 设置缺省中文字体
\usetikzlibrary{arrows, automata, calc, shapes}

\newcommand{\HRule}{\rule{\linewidth}{0.5mm}}
\newcommand{\hytt}[1]{\texttt{\hyphenchar\font=\defaulthyphenchar #1}}
\renewcommand{\algorithmicrequire}{\textbf{Input:}}   
\renewcommand{\algorithmicensure}{\textbf{Output:}}  
% \hyphenation{read-Sym-bol re-ad-Space-Tab-New-line str-Tab}

%\footnotesize
\lstset{ %
  backgroundcolor=\color{white},   % choose the background color; you must add \usepackage{color} or \usepackage{xcolor}
  basicstyle=\ttfamily,            % the size of the fonts that are used for the code
  breakatwhitespace=false,         % sets if automatic breaks should only happen at whitespace
  breaklines=true,                 % sets automatic line breaking
  captionpos=b,                    % sets the caption-position to bottom
  commentstyle=\ttfamily\color{mygreen},    
                                   % comment style
  deletekeywords={},               % if you want to delete keywords from the given language
  escapeinside={},                 % if you want to add LaTeX within your code
  extendedchars=true,              % lets you use non-ASCII characters; for 8-bits encodings only, does not work with UTF-8
  frame=single,                    % adds a frame around the code
  keepspaces=true,                 % keeps spaces in text, useful for keeping indentation of code (possibly needs columns=flexible)
  keywordstyle=\color{blue},       % keyword style
  language=C++,                    % the language of the code
  morekeywords={},                 % if you want to add more keywords to the set
  numbers=left,                    % where to put the line-numbers; possible values are (none, left, right)
  numbersep=5pt,                   % how far the line-numbers are from the code
  numberstyle=\tiny\color{mygray}, % the style that is used for the line-numbers
  rulecolor=\color{black},         % if not set, the frame-color may be changed on line-breaks within not-black text (e.g. comments (green here))
  showspaces=false,                % show spaces everywhere adding particular underscores; it overrides 'showstringspaces'
  showstringspaces=false,          % underline spaces within strings only
  showtabs=false,                  % show tabs within strings adding particular underscores
  stepnumber=1,                    % the step between two line-numbers. If it's 1, each line will be numbered
  stringstyle=\color{mymauve},     % string literal style
  tabsize=2,                       % sets default tabsize to 2 spaces
  title=\lstname                   % show the filename of files included with \lstinputlisting; also try caption instead of title
}

\pagestyle{fancy}
\rhead{page \thepage\ of \pageref{LastPage}}
%\chead{}
\lhead{计算机体系结构}
\cfoot{}

\begin{document}

\title{第二章作业}
\author{计算机1202 \quad 张艺瀚\\学号:20123852}
\maketitle

\thispagestyle{fancy}
%\newpage
\normalsize

\begin{enumerate}
  \item 
    \begin{table}[htbp]
      \centering
      \begin{tabular}{lll}
        \hline
        0000 & xxxxxx & xxxxxx \\
        $\vdots$ & $\vdots$ & $\vdots$ \\
        1110 & xxxxxx & xxxxxx \\
        \hline
        1111 & 000000 & xxxxxx \\
        $\vdots$ & $\vdots$ & $\vdots$ \\
        1111 & 111111 & xxxxxx \\
        \hline
      \end{tabular}
    \end{table}
    单地址指令 $ = \left( 2^4 - 1 - 10 \right) \times 2^6 + 2^6 = 384 $
  \item
    \begin{enumerate}
      \item
        等长二进制编码,平均码长 $ = \lceil \log_2 10 \rceil = 4 $
      \item
        huffman编码(见图\ref{fig: tree}、表\ref{tab: code}) \\
        \begin{figure}
          \begin{center}
            \begin{tikzpicture}
              \tikzstyle{level 1} = [sibling distance = 64mm]
              \tikzstyle{level 2} = [sibling distance = 32mm]
              \tikzstyle{level 3} = [sibling distance = 16mm]
              \tikzstyle{level 4} = [sibling distance = 8mm]
              \tikzstyle{level 5} = [sibling distance = 8mm]
              \node {1.00}
                child{
                  node {0.40}
                  child{
                    node {0.23}
                    child{
                      node {0.11}
                      child{
                        node {0.04}
                        child{
                          node {0.01}
                          edge from parent[left] node {0}
                        }
                        child{
                          node {0.03}
                          edge from parent[right] node {1}
                        }
                        edge from parent[left] node {0}
                      }
                      child{
                        node {0.07}
                        edge from parent[right] node {1}
                      }
                      edge from parent[left] node {0}
                    }
                    child{
                      node {0.12}
                      edge from parent[right] node {1}
                    }
                    edge from parent[left] node {0}
                  }
                  child{
                    node {0.17}
                    child{
                      node {0.08}
                      edge from parent[left] node {0}
                    }
                    child{
                      node {0.09}
                      edge from parent[right] node {1}
                    }
                    edge from parent[right] node {1}
                  }
                  edge from parent[left] node {0}
                }
                child{
                  node {0.6}
                  child{
                    node {0.28}
                    child{
                      node {0.13}
                      edge from parent[left] node {0}
                    }
                    child{
                      node {0.15}
                      edge from parent[right] node {1}
                    }
                    edge from parent[left] node {0}
                  }
                  child{
                    node {0.32}
                    child{
                      node {0.15}
                      edge from parent[left] node {0}
                    }
                    child{
                      node {0.17}
                      edge from parent[right] node {1}
                    }
                    edge from parent[right] node {1}
                  }
                  edge from parent[right] node {1}
                };
            \end{tikzpicture}
            \caption{huffman树\label{fig: tree}}
          \end{center}
        \end{figure}
        \begin{eqnarray*}
          \text{平均码长} &=& (0.17 + 0.15 + 0.15 + 0.13 + 0.12 + 0.09 + 0.08) \times 3 \\
          && + 0.07 \times 4 + (0.03 + 0.01) \times 5 \\
          &=& 3.15
        \end{eqnarray*}
      \item 
        只有2种码长的扩展操作码(见表\ref{tab: code})
        \begin{eqnarray*}
          \text{平均码长} &=& (0.17 + 0.15 + 0.15 + 0.13 + 0.12 + 0.09) \times 3 \\
          && + (0.08 + 0.07 + 0.03 + 0.01) \times 4 \\
          &=& 3.19
        \end{eqnarray*}
    \end{enumerate}
\end{enumerate}

\begin{table}[htbp]
  \centering
  \begin{tabular}{lll}
    \hline
    频度 & huffman编码 & 扩展操作码 \\
    \hline
    0.17 & 111 & 000 \\
    0.15 & 101 & 001 \\
    0.15 & 110 & 010 \\
    0.13 & 100 & 011 \\
    0.12 & 001 & 100 \\
    0.09 & 011 & 101 \\
    \hline
    0.08 & 010 & 1100 \\
    0.07 & 0001 & 1101 \\
    0.03 & 00001 & 1110 \\
    0.01 & 00000 & 1111 \\
    \hline
  \end{tabular}
  \caption{编码表\label{tab: code}}
\end{table}

\end{document}
