\documentclass[11pt]{article}

\usepackage{abstract}
\usepackage{algorithm}
\usepackage{algorithmic}
\usepackage{amsmath}
\usepackage{amssymb}
\usepackage{bm}
\usepackage{caption}
\usepackage{CJKutf8}
\usepackage{color}
\usepackage{enumitem}
\usepackage{epsfig}
\usepackage{fancyhdr}
\usepackage{float}
\usepackage{graphics}
\usepackage{graphicx}
\usepackage{geometry}
\usepackage{indentfirst}
\usepackage{lastpage}
\usepackage{listings}
\usepackage{mathdots}
\usepackage{mathpazo}
\usepackage{multirow}
\usepackage{pstricks-add}
\usepackage{pst-blur}
\usepackage{subcaption}
\usepackage{tikz}
\usepackage{wasysym}
\usepackage{xcolor}
\usepackage[BoldFont,SlantFont,CJKsetspaces,CJKchecksingle]{xeCJK}

\allowdisplaybreaks
\DeclareMathOperator*{\argmin}{argmin}
\definecolor{Blue}{rgb}{1.,0.75,0.8}
\definecolor{mygray}{rgb}{0.5,0.5,0.5}
\definecolor{mygreen}{rgb}{0,0.6,0}
\definecolor{mymauve}{rgb}{0.58,0,0.82}
\pagestyle{empty}
\parindent 2em   %段首缩进
\setlength{\parindent}{2em}
\setCJKmainfont[BoldFont=SimHei]{SimSun}
\setCJKmonofont{SimSun}% 设置缺省中文字体
\usetikzlibrary{arrows, automata, calc, shapes}

\newcommand{\HRule}{\rule{\linewidth}{0.5mm}}
\newcommand{\hytt}[1]{\texttt{\hyphenchar\font=\defaulthyphenchar #1}}
\renewcommand{\algorithmicrequire}{\textbf{Input:}}   
\renewcommand{\algorithmicensure}{\textbf{Output:}}  
% \hyphenation{read-Sym-bol re-ad-Space-Tab-New-line str-Tab}

%\footnotesize
\lstset{ %
  backgroundcolor=\color{white},   % choose the background color; you must add \usepackage{color} or \usepackage{xcolor}
  basicstyle=\ttfamily,            % the size of the fonts that are used for the code
  breakatwhitespace=false,         % sets if automatic breaks should only happen at whitespace
  breaklines=true,                 % sets automatic line breaking
  captionpos=b,                    % sets the caption-position to bottom
  commentstyle=\ttfamily\color{mygreen},    
                                   % comment style
  deletekeywords={},               % if you want to delete keywords from the given language
  escapeinside={},                 % if you want to add LaTeX within your code
  extendedchars=true,              % lets you use non-ASCII characters; for 8-bits encodings only, does not work with UTF-8
  frame=single,                    % adds a frame around the code
  keepspaces=true,                 % keeps spaces in text, useful for keeping indentation of code (possibly needs columns=flexible)
  keywordstyle=\color{blue},       % keyword style
  language=C++,                    % the language of the code
  morekeywords={},                 % if you want to add more keywords to the set
  numbers=left,                    % where to put the line-numbers; possible values are (none, left, right)
  numbersep=5pt,                   % how far the line-numbers are from the code
  numberstyle=\tiny\color{mygray}, % the style that is used for the line-numbers
  rulecolor=\color{black},         % if not set, the frame-color may be changed on line-breaks within not-black text (e.g. comments (green here))
  showspaces=false,                % show spaces everywhere adding particular underscores; it overrides 'showstringspaces'
  showstringspaces=false,          % underline spaces within strings only
  showtabs=false,                  % show tabs within strings adding particular underscores
  stepnumber=1,                    % the step between two line-numbers. If it's 1, each line will be numbered
  stringstyle=\color{mymauve},     % string literal style
  tabsize=2,                       % sets default tabsize to 2 spaces
  title=\lstname                   % show the filename of files included with \lstinputlisting; also try caption instead of title
}

\pagestyle{fancy}
\rhead{page \thepage\ of \pageref{LastPage}}
%\chead{}
\lhead{计算机体系结构}
\cfoot{}

\begin{document}

\title{第一章作业}
\author{计算机1202 \quad 张艺瀚\\学号:20123852}
\maketitle

\thispagestyle{fancy}
%\newpage
\normalsize

\begin{enumerate}
  \item 见表\ref{tab: tab1}
    \begin{table}[htbp]
    \centering  % 表居中
      \begin{tabular}{ll}  % {lccc} 表示各列元素对齐方式,left-l,right-r,center-c
        \hline
        级 &时间(s) \\
        \hline  % \hline 在此行下面画一横线
        1 &$k$ \\
        2 &$Nk/M$ \\
        3 &$N^2k/M^2$ \\
        4 &$N^3k/M^3$ \\
        \hline
      \end{tabular}
    \caption{\label{tab: tab1}}
    \end{table}
  \item 
    \[ CPI = \frac{45000 \times 1 + 75000 \times 2 + 8000 \times 4 + 1500 \times 2}{45000 + 75000 + 8000 + 1500} = 1.78 \]
    \[ MIPS = \frac{CR}{CPI \times 10^6} = \frac{400}{1.78} = 224.72 \]
    \[ t = \frac{45000 \times 1 + 75000 \times 2 + 8000 \times 4 + 1500 \times 2}{400 \times 10^6} = 5.75 \times 10^{-4}s \]
  \item 
    \begin{enumerate}
      \item 
        \begin{eqnarray*}
          10 &=& \dfrac{1}{(1-30\%-30\%-F_3) + \dfrac{30\%}{30} + \dfrac{30\%}{20} + \dfrac{F_3}{10}} \\
          \Longrightarrow F_3 &=& 36\%
        \end{eqnarray*}
      \item 
        \begin{eqnarray*}
          \frac{(1-\sum_i F_i)T_{old}}{(1-\sum_i F_i)T_{old} + \sum_i F_i / S_i \cdot T_{old}} 
          &=& \dfrac{1-30\%-30\%-20\%}{(1-30\%-30\%-20\%) + \dfrac{30\%}{30} + \dfrac{30\%}{20} + \dfrac{20}{10}} \\
          &=& 82\%
        \end{eqnarray*}
    \end{enumerate}
  \item 
    \begin{enumerate}
      \item 见表\ref{tab: tab2}
        \begin{table}[htbp]
        \centering
          \begin{tabular}{ll}
            \hline
            操作&加速比\\
            \hline
            1 & $2/1 = 2$ \\
            2 & $20/15 = 1.33$ \\
            3 & $10/3 = 3.33$ \\
            4 & $4/1 = 4$ \\
            \hline
          \end{tabular}
        \caption{\label{tab: tab2}}
        \end{table}
      \item 见表\ref{tab: tab3}
        \begin{table}[htbp]
        \centering
          \begin{tabular}{ll}
            \hline
            操作&加速比\\
            \hline
            1 & $\dfrac{1}{1-\dfrac{10}{90} + \dfrac{10}{90}/2} = 1.06$ \\
            2 & $\dfrac{1}{1-\dfrac{30}{90} + \dfrac{30}{90}/1.33} = 1.09$ \\
            3 & $\dfrac{1}{1-\dfrac{35}{90} + \dfrac{35}{90}/3.33} = 1.37$ \\
            4 & $\dfrac{1}{1-\dfrac{15}{90} + \dfrac{15}{90}/4} = 1.14$ \\
            \hline
          \end{tabular}
        \caption{\label{tab: tab3}}
        \end{table}
      \item
        \[ SP = \dfrac{1}{\dfrac{10}{90}/2 + \dfrac{30}{90}/1.33 + \dfrac{35}{90}/3.33 + \dfrac{15}{90}/4} = 2.15 \]
    \end{enumerate}
  \item 
    \[ SP = \dfrac{1}{1-40\%+\dfrac{40\%}{10}} = 1.56 \]
\end{enumerate}

\end{document}
